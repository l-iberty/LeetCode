\documentclass{article}
\usepackage{amssymb, amsmath, geometry, ctex}

\begin{document}

\begin{itemize}
	\item 数组长度为$N$,抽样个数为$k$,$i \in [k+1,N]$,\textbf{注意$i$不是数组下标,第$i$个数字是$A[i-1]$}
	\item 每次生成的随机数$j \in [0,i)$,这个区间长度为$i$,按照古典概型,$A[i-1]$被放进长度为$k$的水塘里的概率是$P\{j<k\}=\frac{k}{i}$
	\item 为什么$i$从$k+1$开始,而不能是$k$?因为生成的随机数$j$落在$[0,k-1]$内时会把$A[i-1]$放入水塘。如果$i=k$,水塘内就可能出现两个$A[k-1]$,这是不对的。因此必须保证$A[i-1]$位于水塘之外,即$i>k$,才能避免这种错误。
	\item $A[i-1]$被放进水塘后有可能被后面的数字替换掉,那么$A[i-1]$最终能留在水塘中的概率是多少?对于$A[i-1]$后面的数字$A[i^\prime-1] (i^\prime > i)$,先生成一个随机数$j^\prime \in [0,i^\prime)$,如果$A[i-1]$能被$A[i^\prime-1]$替换,就说明$j^\prime = j$。即,$A[i-1]$被$A[i^\prime-1]$替换掉的概率为$P\{j^\prime=j\}=\frac{1}{i^\prime}$。如果$A[i-1]$最终留在了水塘中,就说明它没有被后面的数字替换掉,概率为:
	\begin{equation*}
		\frac{k}{i}(1-\frac{1}{i+1})(1-\frac{1}{i+2})\cdots(1-\frac{1}{N})=\frac{k}{N}, i \in [k+1,N)
	\end{equation*}
	其中,
	\begin{align*}
		& \frac{k}{i} = A[i-1]\text{被选进水塘的概率} \\
		& 1-\frac{1}{i+1} = A[i-1]\text{不被}A[i]\text{替换掉的概率} \\
		& 1-\frac{1}{i+2} = A[i-1]\text{不被}A[i+1]\text{替换掉的概率} \\
		& \cdots \\
		& 1-\frac{1}{N} = A[i-1]\text{不被}A[N-1]\text{替换掉的概率}
	\end{align*}
	当$i=N$时,也就是要把$A[N-1]$放进水塘,概率是$\frac{k}{N}$。由于后面已经没有数字了,所以不用考虑被替换的情况。
	
	当$i\leqslant k$时,$A[i-1]$位于水塘内,相当于被选进水塘的概率= 1,并且只能被水塘外面的数字,即$A[k]$及其之后的数字替换掉,被替换概率= $(1-\frac{1}{k+1})(1-\frac{1}{k+2})\cdots(1-\frac{1}{N})=\frac{k}{N}$,所以留在水塘中的概率仍是$\frac{k}{N}$。
\end{itemize}


\end{document}